\begin{frame}{Theory}
    Quick crash courses in:
    \vspace{1em}
    \begin{itemize-size}{1em}
        \item \textbf{Stack Resource Policy}
        \item \textbf{Schedulability analysis techniques}
        \item \textbf{The RTIC Framework}
        \item \textbf{Symbolic execution with KLEE}
    \end{itemize-size}
\end{frame}

\begin{frame}{Stack Resource Policy}
    \begin{itemize-size}{1em}
        \item Defines a way for \textbf{sharing resources} in a real-time system
        \item \textbf{Prevents deadlocks}
        \item \textbf{Task} as a \emph{unit of execution}
        \item A task can \textbf{only be blocked once} during its execution
        \item Tasks share a \textbf{single stack} - \emph{very memory efficient}
        \item Works with both \textbf{fixed} and dynamic priorities
    \end{itemize-size}
\end{frame}


\begin{frame}{Schedulability analysis techniques}
    We say that a real-time system is schedulable IFF
    \vspace{1em}
    \begin{block}{Schedulability check}
        \begin{enumerate}
            \item All tasks in the system finish their execution before their deadlines.
            \item The processor utilization (system load) never exceeds 100\%
        \end{enumerate}
    \end{block}
\end{frame}

\begin{frame}{Schedulability anaysis techniques}
    \textbf{Question:} how to determine if the system adheres to \textbf{1.} and \textbf{2.}?

\end{frame}

\begin{frame}{Schedulability analysis techniques}
    \begin{block}{Answer}
        \begin{itemize-size}{1em}
            \item To check if deadlines are met
            \begin{itemize-size}{1em}
                \vspace{0.5em}
                \item Check if \textbf{worst-case response time} finishes before the deadline
                \item "In the worst case, will the task respond before its deadline?"
            \end{itemize-size}
            \item To check if system is not overloaded
            \begin{itemize-size}{1em}
                \vspace{0.5em}
                \item Determine the worst-case load each task has on the system
                \item Add them together and make sure it does not exceed 100\%
            \end{itemize-size}
        \end{itemize-size} 
    \end{block}
\end{frame}

\begin{frame}{Schedulability analysis techniques}
    The \textbf{worst-case response time (WCRT)} $R_i$ of a \textbf{task} can be calculated with
    the following recurrence equation:
    \begin{equation}
        \begin{cases}
            R_{i}(0) &= C_i + B_i \\
            R_{i}(s) &= C_i + B_i + \sum\limits_{h: P_h > P_i} \left\lceil \frac{R_{i}(s-1)}{T_h} \right\rceil.
        \end{cases}
    \end{equation}
    Where $C_i$ is the \emph{worst-case execution time} and $B_i$ the
    \emph{blocking time} on the task.
    \vspace{1em}

    \textbf{Source:} \emph{Hard Real-time Computing Systems} by Giorgio Buttazzo.
\end{frame}

\begin{frame}{Schedulability analysis techniques}
    The \textbf{worst-case system load} of the system can be calculated as:
    \begin{equation}
        \rho = \sum^{n}_{i=1} \frac{C_i}{T_i}
    \end{equation}
    Where $C_i$ is the \emph{worst-case execution time} and $T_i$ the
    \emph{period/inter-arrival} of the task.
    \vspace{1em}

    \textbf{Source:} \emph{Hard Real-time Computing Systems} by Giorgio Buttazzo.
\end{frame}

\begin{frame}{Schedulability analysis techniques}
    What is the difference between WCET and WCRT?
    \vspace{1em}

    \begin{block}{Worst-case Execution Time}
        The maximum possible execution time of a task, without any
        interruptions from e.g. higher priority tasks.
    \end{block}

    \vspace{1em}

    \begin{block}{Worst-case Response Time}
        The maximum possible execution time of a task, with all
        possible interruptions such as blocking time considered.
    \end{block}
\end{frame}

\begin{frame}{Schedulability analysis techniques}
    \begin{block}{Blocking time}
        Longest time a \emph{higher priority task} can be blocked from accessing a
        shared resource by a \emph{lower priority task}
    \end{block}
\end{frame}




\begin{frame}{The RTIC Framework}
    \begin{columns}
        \column{0.45\textwidth}
        \begin{itemize-size}{1em}
            \item \textbf{Framework to build real-time systems in the Rust language}
            \item \textbf{For ARM Cortex-M devices}
            \item \textbf{Fixed-priority preemptive scheduling}
            \item \textbf{Minimal scheduling overhead}
            \item \textbf{Memory sharing following the Stack Resource Policy}
        \end{itemize-size}

        \column{0.55\textwidth}
        \begin{figure}
            \centering
            \includegraphics[scale=0.35]{pictures/RTIC.png}
            \caption{The RTIC logo}
        \end{figure}
    \end{columns}
\end{frame}

\begin{frame}{The RTIC Framework}
    \begin{columns}
        \column{0.45\textwidth}
        \begin{itemize-size}{1em}
            \item Framework to build real-time systems in the Rust language
            \item For ARM Cortex-M devices
            \item \textbf{\textcolor{green!70}{Fixed-priority preemptive scheduling}}
            \item Minimal scheduling overhead
            \item \textbf{\textcolor{green!70}{Memory sharing following the
                Stack Resource Policy}}
        \end{itemize-size}

        \column{0.55\textwidth}
        \begin{figure}
            \centering
            \includegraphics[scale=0.35]{pictures/RTIC.png}
            \caption{The RTIC logo}
        \end{figure}
    \end{columns}
\end{frame}


\begin{frame}{The RTIC Framework}
    \begin{columns}
        \column{0.45\textwidth}
        \begin{itemize-size}{1em}
            \item \textbf{The user defines:}
            \begin{itemize-size}{1em}
                \item Initialization function
                \item \emph{User tasks}
                \item Idle function
            \end{itemize-size}
            \item \textbf{When building the application:}
            \begin{itemize-size}{1em}
                \item Expand application with support code
                and extra details
            \end{itemize-size}
        \end{itemize-size}

        \column{0.55\textwidth}
        \begin{figure}
            \centering
            \includegraphics[scale=0.35]{pictures/RTIC.png}
            \caption{The RTIC logo}
        \end{figure}
    \end{columns}
\end{frame}

\begin{frame}{KLEE - Symbolic Execution Engine}
    \begin{columns}
        \column{0.45\textwidth}
        \begin{itemize-size}{1em}
            \item \textbf{A way to test programs}
            \item \textbf{Symbolic instead of concrete}
            \begin{itemize-size}{1em}
                \item E.g. fuzz testing works on concrete values
            \end{itemize-size}
        \item \textbf{Explores paths in the program to find errors}
        \item \textbf{Each path will generate a concrete test vector}
        \item \textbf{Path explosion problem!}
        \end{itemize-size}

        \column{0.55\textwidth}
        \begin{figure}
            \centering
            \includegraphics[scale=0.55]{pictures/klee.png}
        \end{figure}
    \end{columns}
\end{frame}

\begin{frame}{KLEE - Example}
\lstinputlisting[
    language={rust},
]{../code/klee_function.rs}
\end{frame}

\begin{frame}{KLEE - Example}
\begin{figure}
    \centering
    \begin{tikzpicture}[node distance=1.5cm]
        \node (function) [orangerectangle] {function(input)};
        \node (first) [draw, diamond, below of=function, aspect=2, yshift=-5mm] {$input < 2000$};
        \node (complete_left) [redrectangle, below of=first, aspect=2, xshift=-2.5cm, fill=black!50!green!60] {complete};
        \node (increment) [redrectangle, below of=first, xshift=2.5cm, fill=blue!30] {$input+ 1000$};
        \node (1234) [draw, diamond, below of=increment, aspect=2] {$input == 1234$};
        \node (panic) [redrectangle, below of=1234, aspect=2, xshift=2.5cm, fill=red!60] {panic};
        \node (complete_right) [redrectangle, below of=1234, aspect=2, xshift=-2.5cm, fill=black!50!green!60] {complete};

        \draw [arrow] (function) -- (first);
        \draw [arrow] (first) -- node[xshift=-0.6cm]{false} (complete_left);
        \draw [arrow] (first) -- node[xshift=0.6cm]{true} (increment);
        \draw [arrow] (increment) -- (1234);
        \draw [arrow] (1234) -- node[xshift=-0.6cm]{false} (complete_right);
        \draw [arrow] (1234) -- node[xshift=0.6cm]{true} (panic);
    \end{tikzpicture}
    \caption{Simplified visual representation of KLEE running on previous function.}
\end{figure}
\end{frame}

