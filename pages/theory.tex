\begin{frame}{Theory}
    \begin{itemize-size}{1em}
        \item \textbf{RTIC}
        \item \textbf{Symbolic execution with KLEE}
        \item \textbf{Schedulability analysis techniques}
    \end{itemize-size}
\end{frame}

\begin{frame}{The RTIC Framework}
    \begin{columns}
        \column{0.45\textwidth}
        \begin{itemize-size}{1em}
            \item \textbf{Framework to build real-time systems in the Rust language}
            \item \textbf{Tasks as a unit of execution}
            \item \textbf{Fixed-priority preemptive scheduling}
            \item \textbf{Minimal scheduling overhead}
            \item \textbf{Memory sharing following the Stack Resource Policy}
            \item \textbf{For ARM Cortex-M devices}
        \end{itemize-size}

        \column{0.55\textwidth}
        \begin{figure}
            \centering
            \includegraphics[scale=0.35]{pictures/RTIC.png}
            \caption{The RTIC logo}
        \end{figure}
    \end{columns}
\end{frame}

\begin{frame}{KLEE - Symbolic Execution Engine}
    \begin{columns}
        \column{0.45\textwidth}
        \begin{itemize-size}{1em}
            \item \textbf{A way to test programs}
            \item \textbf{Symbolic instead of concrete}
            \begin{itemize-size}{1em}
                \item E.g. fuzz testing works on concrete values 
            \end{itemize-size}
        \item \textbf{Explores paths in the program to find errors}
        \item \textbf{Each path will generate a concrete test vector}
        \item \textbf{Path explosion problem!}
        \end{itemize-size}

        \column{0.55\textwidth}
        \begin{figure}
            \centering
            \includegraphics[scale=0.35]{example-image-a}
        \end{figure}
    \end{columns}
\end{frame}

\begin{frame}{KLEE - Example}
\lstinputlisting[
    language={rust},
]{../code/klee_function.rs}
\end{frame}

\begin{frame}{KLEE - Example}
\begin{figure}
    \centering
    \begin{tikzpicture}[node distance=1.5cm]
        \node (function) [orangerectangle] {function(input)};
        \node (first) [draw, diamond, below of=function, aspect=2, yshift=-5mm] {$input < 2000$};
        \node (complete_left) [redrectangle, below of=first, aspect=2, xshift=-2.5cm, fill=black!50!green!60] {complete};
        \node (increment) [redrectangle, below of=first, xshift=2.5cm, fill=blue!30] {$input+ 1000$};
        \node (1234) [draw, diamond, below of=increment, aspect=2] {$input == 1234$};
        \node (panic) [redrectangle, below of=1234, aspect=2, xshift=2.5cm, fill=red!60] {panic};
        \node (complete_right) [redrectangle, below of=1234, aspect=2, xshift=-2.5cm, fill=black!50!green!60] {complete};

        \draw [arrow] (function) -- (first);
        \draw [arrow] (first) -- node[xshift=-0.6cm]{false} (complete_left);
        \draw [arrow] (first) -- node[xshift=0.6cm]{true} (increment);
        \draw [arrow] (increment) -- (1234);
        \draw [arrow] (1234) -- node[xshift=-0.6cm]{false} (complete_right);
        \draw [arrow] (1234) -- node[xshift=0.6cm]{true} (panic);
    \end{tikzpicture}
    \caption{Simplified visual representation of KLEE running on previous function.}
\end{figure}
\end{frame}

\begin{frame}{Schedulability analysis techniques}
    We say that a system is schedulable IFF

    \begin{block}{Schedulability check}
        \begin{enumerate}
            \item All tasks in the system finish their execution before their deadlines.
            \item The processor utilization (system load) never exceeds 100\% 
        \end{enumerate}
    \end{block}
    
\end{frame}
