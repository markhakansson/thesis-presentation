\begin{frame}{Implementation}
    In order to run a schedulability analysis on the RTIC application,
    RAUK needs to do the following
    \begin{itemize-size}{1em}
        \item Measure the WCET of all tasks
        \item Record a trace of the tasks in order to
        calculate the blocking time
    \end{itemize-size}     
\end{frame}

\begin{frame}{Implementation}
    \begin{block}{Generate test vectors}
        \begin{itemize-size}{1em}
            \item We build the RTIC application such that KLEE can run on it
            \item \textbf{IN THEORY:} We can generate test vectors targeting all possible
            execution paths on the user tasks using KLEE
            \item \textbf{ASSUMPTION:} At least one test vector generated for a task, should
            result in the WCET of that task!
        \end{itemize-size}  
    \end{block}
\end{frame}

\begin{frame}{Implementation}
    \begin{block}{Replay the test vectors}
        \begin{itemize-size}{1em}
            \item Build the application such that we can test it on the target hardware
            \item Each test vector will be executed on the target hardware, by overwriting
            memory addresses and registers as necessary
            \item A trace of each test vector will be recorded using breakpoints
            \begin{itemize-size}{1em}
                \item Records when a task starts/ends
                \item Records when a task claims/releases a resource
            \end{itemize-size}
        \end{itemize-size}  
    \end{block}
\end{frame}

\begin{frame}{Implementation}
    \begin{block}{Schedulability analysis}
        After recording the traces RAUK can
        \begin{itemize-size}{1em}
            \item Calculate the WCET for each trace
            \item Determine blocking times on all tasks using the traces
            \item Calculate WCRT for each trace
            \item Determine wheter the system is schedulable or not
        \end{itemize-size}  
    \end{block}
\end{frame}
